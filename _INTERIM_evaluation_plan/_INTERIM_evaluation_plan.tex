\chapter{Evaluation Plan}

The evaluation of this project will combine quantitative technical metrics and qualitative assessments. The aim is to evaluate the verification framework's functionality, scalability, and real-world applicability.

\paragraph{Technical Evaluation}
The technical evaluation will measure the method's accuracy, efficiency, and scalability. This includes:
\begin{itemize}
    \item \textbf{Verification Accuracy}: The framework will be tested to ensure it provides reliable and accurate verification results. This involves analyzing the over-approximation of reachable states in LNNs whilst avoiding excessive conservatism.
    \item \textbf{Scalability Testing}: The framework's performance will be assessed across LNNs of varying sizes and complexities to determine its computational efficiency in terms of runtime and memory consumption.
    \item \textbf{Benchmarking}: The developed method will be compared to existing verification techniques, such as zonotope-based and star-based reachability approaches, to highlight its strengths and identify any limitations.
\end{itemize}

\paragraph{Qualitative Evaluation}
The qualitative evaluations will assess usability and relevance in practical applications. This includes:
\begin{itemize}
    \item \textbf{User Accessibility}: The framework will be assessed for ease of use, including the clarity of its documentation, simplicity of installation, and integration into existing workflows.
    \item \textbf{Real-World Applicability}: The method will be applied to LNNs in practical tasks such as control systems and time-series prediction (or simulations, if this proves difficult) to evaluate its effectiveness in solving real-world problems.
    \item \textbf{Adaptability}: Its ability to handle various LNN architectures, including different activation functions and parameter configurations, will be evaluated.
\end{itemize}

\paragraph{Case Studies}
The evaluation will include targeted case studies to test the method under specific scenarios, such as verifying robustness against adversarial inputs, ensuring safety in control systems, and analyzing stability under input variations.

\paragraph{Iterative Refinement}
An iterative approach will be taken to evaluation, incorporating periodic results from technical tests, user experience and supervisor feedback to refine the verification framework.