\chapter{Ethical Issues}

This project is a software-based initiative and does not involve experiments on any living beings. Additionally, it poses no significant physical or environmental safety risks. No personal or sensitive data is collected or utilized during the course of this work. All datasets used for experimental purposes will be sourced from publicly available open-source repositories. In addition, care will be taken to ensure all datasets are appropriately anonymised and used in compliance with relevant data protection regulations.

There are no immediate military/high-risk applications envisioned for this project. However, liquid neural networks are highly adaptable and capable of handling complex, dynamic tasks. Thus, they could be misused in fields such as surveillance, privacy-invasive technologies, and military systems. Developing robust verification methods would increase the reliability of LNNs, which carries the risk of increasing their impact in such areas. However, since this is a theoretical verification project, this risk is no greater than that of other general-purpose machine learning research.

Ensuring that the verification process is transparent, reproducible, and understandable is important. If verification methods become too opaque, it could create barriers to accountability and limit their ethical deployment.

Software development for this project may require use of open-source libraries. Where relevant, credit will be given to authors. The appropriate academic license will be obtained if any proprietary software is needed for development/experimentation. Proprietary software will not be distributed in the final framework produced.